% Options for packages loaded elsewhere
\PassOptionsToPackage{unicode}{hyperref}
\PassOptionsToPackage{hyphens}{url}
\documentclass[
  11pt,
  ignorenonframetext,
]{beamer}
\newif\ifbibliography
\usepackage{pgfpages}
\setbeamertemplate{caption}[numbered]
\setbeamertemplate{caption label separator}{: }
\setbeamercolor{caption name}{fg=normal text.fg}
\beamertemplatenavigationsymbolsempty
% remove section numbering
\setbeamertemplate{part page}{
  \centering
  \begin{beamercolorbox}[sep=16pt,center]{part title}
    \usebeamerfont{part title}\insertpart\par
  \end{beamercolorbox}
}
\setbeamertemplate{section page}{
  \centering
  \begin{beamercolorbox}[sep=12pt,center]{section title}
    \usebeamerfont{section title}\insertsection\par
  \end{beamercolorbox}
}
\setbeamertemplate{subsection page}{
  \centering
  \begin{beamercolorbox}[sep=8pt,center]{subsection title}
    \usebeamerfont{subsection title}\insertsubsection\par
  \end{beamercolorbox}
}
% Prevent slide breaks in the middle of a paragraph
\widowpenalties 1 10000
\raggedbottom
\AtBeginPart{
  \frame{\partpage}
}
\AtBeginSection{
  \ifbibliography
  \else
    \frame{\sectionpage}
  \fi
}
\AtBeginSubsection{
  \frame{\subsectionpage}
}
\usepackage{iftex}
\ifPDFTeX
  \usepackage[T1]{fontenc}
  \usepackage[utf8]{inputenc}
  \usepackage{textcomp} % provide euro and other symbols
\else % if luatex or xetex
  \usepackage{unicode-math} % this also loads fontspec
  \defaultfontfeatures{Scale=MatchLowercase}
  \defaultfontfeatures[\rmfamily]{Ligatures=TeX,Scale=1}
\fi
\usepackage{lmodern}
\usetheme[]{Madrid}
\usecolortheme[]{beaver}
\ifPDFTeX\else
  % xetex/luatex font selection
\fi
% Use upquote if available, for straight quotes in verbatim environments
\IfFileExists{upquote.sty}{\usepackage{upquote}}{}
\IfFileExists{microtype.sty}{% use microtype if available
  \usepackage[]{microtype}
  \UseMicrotypeSet[protrusion]{basicmath} % disable protrusion for tt fonts
}{}
\makeatletter
\@ifundefined{KOMAClassName}{% if non-KOMA class
  \IfFileExists{parskip.sty}{%
    \usepackage{parskip}
  }{% else
    \setlength{\parindent}{0pt}
    \setlength{\parskip}{6pt plus 2pt minus 1pt}}
}{% if KOMA class
  \KOMAoptions{parskip=half}}
\makeatother
\setlength{\emergencystretch}{3em} % prevent overfull lines
\providecommand{\tightlist}{%
  \setlength{\itemsep}{0pt}\setlength{\parskip}{0pt}}
\usepackage{bookmark}
\IfFileExists{xurl.sty}{\usepackage{xurl}}{} % add URL line breaks if available
\urlstyle{same}
\hypersetup{
  pdftitle={Beyond Expected Goals: A Probabilistic Framework for Shot Occurrences in Soccer},
  pdfauthor={Tianshu Feng, Jonathan Pipping, \& Paul Sabin},
  hidelinks,
  pdfcreator={LaTeX via pandoc}}

\title[Beyond Expected Goals]{Beyond Expected Goals: A Probabilistic
Framework for Shot Occurrences in Soccer}
\author[Feng, T., Pipping, J., \& Sabin, P.]{Tianshu Feng, Jonathan
Pipping, \& Paul Sabin}
\date{2025-07-29}
\institute[UPenn]{University of Pennsylvania}

\begin{document}
\frame{\titlepage}

\begin{frame}{What are Expected Goals (xG)?}
\protect\phantomsection\label{what-are-expected-goals-xg}
\begin{itemize}
\tightlist
\item
  Expected Goals (xG) estimates the probability that a shot is scored
\item
  Estimated by an XGBoost model trained on historical shot data
\item
  Depends on factors like distance from goal, angle to goal, shot type,
  and player positions
\item
  Often used to measure the quality of a chance or a team's performance
  across a game
\end{itemize}
\end{frame}

\begin{frame}{Limitations of xG}
\protect\phantomsection\label{limitations-of-xg}
\begin{itemize}
\tightlist
\item
  Selection bias: xG is only recorded for shots we observe!
\item
  Better shooters are over-represented in the data
\item
  Significant chances without a shot event aren't recorded
\item
  Misses opportunities where players should have shot but didn't
\end{itemize}
\end{frame}

\begin{frame}{Examples of xG Limitations}
\protect\phantomsection\label{examples-of-xg-limitations}
\emph{Video of big chance with no shot}

\emph{Video of multiple shots on one attack}
\end{frame}

\begin{frame}{Methods}
\protect\phantomsection\label{methods}
\begin{itemize}
\tightlist
\item
  Statistical approach
\item
  Data processing steps

  \begin{itemize}
  \tightlist
  \item
    \(\theta\)
  \end{itemize}
\item
  Model specifications
\end{itemize}
\end{frame}

\begin{frame}{Results}
\protect\phantomsection\label{results}
\begin{itemize}
\tightlist
\item
  Key findings
\item
  Statistical significance
\item
  Practical implications
\end{itemize}
\end{frame}

\begin{frame}{Conclusions}
\protect\phantomsection\label{conclusions}
\begin{itemize}
\tightlist
\item
  Summary of main points
\item
  Future work
\item
  Questions and discussion
\end{itemize}
\end{frame}

\begin{frame}{Thank You}
\protect\phantomsection\label{thank-you}
Questions?
\end{frame}

\end{document}
