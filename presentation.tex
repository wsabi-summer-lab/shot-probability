% Options for packages loaded elsewhere
\PassOptionsToPackage{unicode}{hyperref}
\PassOptionsToPackage{hyphens}{url}
\documentclass[
  11pt,
  ignorenonframetext,
]{beamer}
\newif\ifbibliography
\usepackage{pgfpages}
\setbeamertemplate{caption}[numbered]
\setbeamertemplate{caption label separator}{: }
\setbeamercolor{caption name}{fg=normal text.fg}
\beamertemplatenavigationsymbolsempty
% remove section numbering
\setbeamertemplate{part page}{
  \centering
  \begin{beamercolorbox}[sep=16pt,center]{part title}
    \usebeamerfont{part title}\insertpart\par
  \end{beamercolorbox}
}
\setbeamertemplate{section page}{
  \centering
  \begin{beamercolorbox}[sep=12pt,center]{section title}
    \usebeamerfont{section title}\insertsection\par
  \end{beamercolorbox}
}
\setbeamertemplate{subsection page}{
  \centering
  \begin{beamercolorbox}[sep=8pt,center]{subsection title}
    \usebeamerfont{subsection title}\insertsubsection\par
  \end{beamercolorbox}
}
% Prevent slide breaks in the middle of a paragraph
\widowpenalties 1 10000
\raggedbottom
\AtBeginPart{
  \frame{\partpage}
}
\AtBeginSection{
  \ifbibliography
  \else
    \frame{\sectionpage}
  \fi
}
\AtBeginSubsection{
  \frame{\subsectionpage}
}
\usepackage{iftex}
\ifPDFTeX
  \usepackage[T1]{fontenc}
  \usepackage[utf8]{inputenc}
  \usepackage{textcomp} % provide euro and other symbols
\else % if luatex or xetex
  \usepackage{unicode-math} % this also loads fontspec
  \defaultfontfeatures{Scale=MatchLowercase}
  \defaultfontfeatures[\rmfamily]{Ligatures=TeX,Scale=1}
\fi
\usepackage{lmodern}
\usetheme[]{Madrid}
\usecolortheme[]{beaver}
\ifPDFTeX\else
  % xetex/luatex font selection
\fi
% Use upquote if available, for straight quotes in verbatim environments
\IfFileExists{upquote.sty}{\usepackage{upquote}}{}
\IfFileExists{microtype.sty}{% use microtype if available
  \usepackage[]{microtype}
  \UseMicrotypeSet[protrusion]{basicmath} % disable protrusion for tt fonts
}{}
\makeatletter
\@ifundefined{KOMAClassName}{% if non-KOMA class
  \IfFileExists{parskip.sty}{%
    \usepackage{parskip}
  }{% else
    \setlength{\parindent}{0pt}
    \setlength{\parskip}{6pt plus 2pt minus 1pt}}
}{% if KOMA class
  \KOMAoptions{parskip=half}}
\makeatother
\setlength{\emergencystretch}{3em} % prevent overfull lines
\providecommand{\tightlist}{%
  \setlength{\itemsep}{0pt}\setlength{\parskip}{0pt}}
\usepackage{bookmark}
\IfFileExists{xurl.sty}{\usepackage{xurl}}{} % add URL line breaks if available
\urlstyle{same}
\hypersetup{
  pdftitle={Beyond Expected Goals: A Probabilistic Framework for Shot Occurrences in Soccer},
  pdfauthor={Tianshu Feng, Jonathan Pipping, and Paul Sabin},
  hidelinks,
  pdfcreator={LaTeX via pandoc}}

\title[Beyond Expected Goals]{Beyond Expected Goals: A Probabilistic
Framework for Shot Occurrences in Soccer}
\author[Feng, T., Pipping, J., \& Sabin, P.]{Tianshu Feng, Jonathan
Pipping, and Paul Sabin}
\date{\today}
\institute[UPenn]{University of Pennsylvania}
\logo{\includegraphics[height=1cm]{wharton.png}}

\begin{document}
\frame{\titlepage}

\begin{frame}{What are Expected Goals (xG)?}
\protect\phantomsection\label{what-are-expected-goals-xg}
\begin{itemize}
\tightlist
\item
  Expected Goals (xG) estimates the probability that a shot is scored
\item
  Estimated by an XGBoost model trained on historical shot data
\item
  Depends on factors like distance from goal, angle to goal, shot type,
  and player positions
\item
  Often used to measure the quality of a chance or a team's performance
  across a game
\end{itemize}
\end{frame}

\begin{frame}{Limitations of xG}
\protect\phantomsection\label{limitations-of-xg}
\begin{itemize}
\tightlist
\item
  Selection bias: xG is only recorded for shots we observe!
\item
  Better shooters are over-represented in the data
\item
  Significant chances without a shot event aren't recorded
\item
  Misses opportunities where players should have shot but didn't
\end{itemize}
\end{frame}

\begin{frame}{Examples of xG Limitations}
\protect\phantomsection\label{examples-of-xg-limitations}
\emph{Video of big chance with no shot}

\emph{Video of multiple shots on one attack}
\end{frame}

\begin{frame}{Methods}
\protect\phantomsection\label{methods}
\begin{itemize}
\tightlist
\item
  Statistical approach
\item
  Data processing steps
\item
  Model specifications
\end{itemize}
\end{frame}

\begin{frame}[fragile]{Statistical Approach}
\protect\phantomsection\label{statistical-approach}
\begin{itemize}
\tightlist
\item
  \texttt{xShot}: the probability that a shot occurs in the next second
\item
  Build a model to estimate \texttt{xShot} based on features from
  tracking data
\item
  Also build our own version of \texttt{xG} model using the same
  features on observed shots
\item
  Estimate the probability of goal as
  \(P(goal) = P(shot) \cdot P(goal | shot)\)
\end{itemize}
\end{frame}

\begin{frame}[fragile]{Data Processing}
\protect\phantomsection\label{data-processing}
\begin{itemize}
\tightlist
\item
  Remove games where no shots are recorded
\item
  Only keep frames where the ball is in play and a team has clear
  possession
\item
  Linearly interpolate ball positions to fill in missing frames
\item
  \texttt{attack}: Index of the attack the current frame is on (0 if it
  is not on an attack)

  \begin{itemize}
  \tightlist
  \item
    Start with the attacking team gaining possession in their attacking
    third
  \item
    End with the defending team regaining possession or the ball is out
    of their attacking third
  \item
    Only keep frames with \texttt{attack\ \textgreater{}\ 0}
  \end{itemize}
\end{itemize}
\end{frame}

\begin{frame}{Data Processing}
\protect\phantomsection\label{data-processing-1}
\begin{itemize}
\tightlist
\item
  Rotate the coordinates \(180^\circ\) around the center point for
  frames where the team attacks from right to left to unify the
  attacking directions and make all \(x\)-coordinates positive
\item
  Use a polar coordinate system centered on the goal for the ball

  \begin{itemize}
  \tightlist
  \item
    \(r_{ball}\) and \(\theta_{ball}\) represent the distance and angle
    of the ball from the goal
  \item
    Keep the \(z\)-coordinate and compute the speed of the ball
  \end{itemize}
\item
  Use a polar coordinate system centered on the ball for each player

  \begin{itemize}
  \tightlist
  \item
    Choose the 5 closest offense teammates and non-GK defenders to the
    ball as features
  \item
    Keep goalkeeper positions as a separate feature
  \end{itemize}
\end{itemize}
\end{frame}

\begin{frame}[fragile]{Data Processing}
\protect\phantomsection\label{data-processing-2}
\begin{itemize}
\tightlist
\item
  \texttt{openGoal}: Percentage of the goal that is open from the ball's
  position

  \begin{itemize}
  \tightlist
  \item
    Simplify every defender as a circle with a radius of 0.75 m
  \item
    Compute the two tangent lines from the ball to every defender in
    front of the ball their intersection points with the goal line
  \item
    Calculate the length of the open goal as the length of goal not
    covered by segments formed by the intersection points
  \end{itemize}
\end{itemize}
\end{frame}

\begin{frame}[fragile]{Model Specifications}
\protect\phantomsection\label{model-specifications}
\begin{itemize}
\tightlist
\item
  Trained on all tracking data of 2022-2025 Premier League seasons
\item
  Use a 5-fold cross-validation to evaluate both \texttt{xG} and
  \texttt{xShot} XGBoost models
\item
  Choose log loss as the evaluation metric
\end{itemize}
\end{frame}

\begin{frame}{Results}
\protect\phantomsection\label{results}
\begin{itemize}
\tightlist
\item
  Key findings
\item
  Statistical significance
\item
  Practical implications
\end{itemize}
\end{frame}

\begin{frame}{Conclusions}
\protect\phantomsection\label{conclusions}
\begin{itemize}
\tightlist
\item
  Summary of main points
\item
  Future work
\item
  Questions and discussion
\end{itemize}
\end{frame}

\begin{frame}{Thank You}
\protect\phantomsection\label{thank-you}
Questions?
\end{frame}

\end{document}
